\section{Related Work} 
\label{sec:soa}

%\cite{ElidanInferenceLess} This paper focuses on efficient parameter learning for missing values given a lower-bound to log-likelihood function.
%\cite{Elidan2011} I dont have much impression on this paper, I checked it, it is about a variation of EM algorithm, so I guess it is not that important.
%\cite{ElidanCopula} This is the paper for the CBN framework. It proposed the most important framework, but for structure learning, they have nothing special.
%\cite{Friedman99} This paper proposed a constraint based structure learning but it has a preconfiguration of the maximal number of parents.
 %\cite{pc_spirtes} This is the paper for the PC algorithm.  
%\cite{heckerman95} This is a bible-like paper for BNs I guess.
% The introduction of 
Copula Bayesian networks  (CBNs), introduced by \cite{ElidanCopula}, are a new and powerful model into
the domain of causal discovery. The main advantage of such an approach is the high flexibility to represent
multivariate distribution by using a graph-structured Copula model.
%introducing Copula functions in graphical model. 
Preliminary work has already overcome some standard problems like
% the problem of 
efficient parameter learning of CBN for datasets with missing values \cite{ElidanInferenceLess}, where a lower bound for the log-likelihood function is proposed.
%
However, many of the algorithms to infer the structure or the parameters of BN rely on mechanisms that have
already been used for genuine BNs induction, like the constraint based structure learning \cite{Friedman99}. %Such 
%approaches have already been improved to, e.g., allow for any number of parents in the final model [INSERT REF].
The two most common algorithms for the structure inference of CBNs are PC algorithm and bayesian information criterion (BIC) score based heuristic method.
The first, PC algorithm \cite{pc_spirtes}, learns the network structure using a constraint based method.
Starting from a completely connected graph, edges are removed if the corresponding 
independences are given.
This, of course, requires a large amount of conditional independence tests, i.e., for each subset of nodes in 
the graph. 
Second, 
%And 
the BIC score based heuristic method \cite{schwarz78} generates the network structure 
in a greedy way, so the optimal structure is not guaranteed. %thus the search space i exponentially large.
%
Both PC and BIC score based heuristic 
methods will estimate parameters based on their learned structure using any traditional estimator, like MLE or
MAP.
%
In contrast, we suggest the use of the partial inverse correlation matrix theorem
which largely reduces the amount of conditional independence tests so that the learning
process is extremely fast. % than other traditional methods.
Moreover, the parameter learning in our PICM-CBN algorithm does not require any explicit structure. %to be learned 
% firstly, 
Instead, the correlation matrix (as we assume Gaussian Copulas) from parameter learning can be  
used further as the input for the structure learning mitigating the computational complexity.
%
% Therefore, we gain a speedup avoiding a big number of conditional 
% independence tests. 
Moreover, even with a large amount of missing observations, the estimated correlation 
matrix for Gaussian Copula still results in a precise structure inference.
% 
% is still enough for a precise inference of structure. 
% This differs impressively 
% from other structure learning algorithms where significant statistics from data is required.

